% Options for packages loaded elsewhere
\PassOptionsToPackage{unicode}{hyperref}
\PassOptionsToPackage{hyphens}{url}
%
\documentclass[
]{article}
\usepackage{amsmath,amssymb}
\usepackage{iftex}
\ifPDFTeX
  \usepackage[T1]{fontenc}
  \usepackage[utf8]{inputenc}
  \usepackage{textcomp} % provide euro and other symbols
\else % if luatex or xetex
  \usepackage{unicode-math} % this also loads fontspec
  \defaultfontfeatures{Scale=MatchLowercase}
  \defaultfontfeatures[\rmfamily]{Ligatures=TeX,Scale=1}
\fi
\usepackage{lmodern}
\ifPDFTeX\else
  % xetex/luatex font selection
\fi
% Use upquote if available, for straight quotes in verbatim environments
\IfFileExists{upquote.sty}{\usepackage{upquote}}{}
\IfFileExists{microtype.sty}{% use microtype if available
  \usepackage[]{microtype}
  \UseMicrotypeSet[protrusion]{basicmath} % disable protrusion for tt fonts
}{}
\makeatletter
\@ifundefined{KOMAClassName}{% if non-KOMA class
  \IfFileExists{parskip.sty}{%
    \usepackage{parskip}
  }{% else
    \setlength{\parindent}{0pt}
    \setlength{\parskip}{6pt plus 2pt minus 1pt}}
}{% if KOMA class
  \KOMAoptions{parskip=half}}
\makeatother
\usepackage{xcolor}
\usepackage[margin=1in]{geometry}
\usepackage{color}
\usepackage{fancyvrb}
\newcommand{\VerbBar}{|}
\newcommand{\VERB}{\Verb[commandchars=\\\{\}]}
\DefineVerbatimEnvironment{Highlighting}{Verbatim}{commandchars=\\\{\}}
% Add ',fontsize=\small' for more characters per line
\usepackage{framed}
\definecolor{shadecolor}{RGB}{248,248,248}
\newenvironment{Shaded}{\begin{snugshade}}{\end{snugshade}}
\newcommand{\AlertTok}[1]{\textcolor[rgb]{0.94,0.16,0.16}{#1}}
\newcommand{\AnnotationTok}[1]{\textcolor[rgb]{0.56,0.35,0.01}{\textbf{\textit{#1}}}}
\newcommand{\AttributeTok}[1]{\textcolor[rgb]{0.13,0.29,0.53}{#1}}
\newcommand{\BaseNTok}[1]{\textcolor[rgb]{0.00,0.00,0.81}{#1}}
\newcommand{\BuiltInTok}[1]{#1}
\newcommand{\CharTok}[1]{\textcolor[rgb]{0.31,0.60,0.02}{#1}}
\newcommand{\CommentTok}[1]{\textcolor[rgb]{0.56,0.35,0.01}{\textit{#1}}}
\newcommand{\CommentVarTok}[1]{\textcolor[rgb]{0.56,0.35,0.01}{\textbf{\textit{#1}}}}
\newcommand{\ConstantTok}[1]{\textcolor[rgb]{0.56,0.35,0.01}{#1}}
\newcommand{\ControlFlowTok}[1]{\textcolor[rgb]{0.13,0.29,0.53}{\textbf{#1}}}
\newcommand{\DataTypeTok}[1]{\textcolor[rgb]{0.13,0.29,0.53}{#1}}
\newcommand{\DecValTok}[1]{\textcolor[rgb]{0.00,0.00,0.81}{#1}}
\newcommand{\DocumentationTok}[1]{\textcolor[rgb]{0.56,0.35,0.01}{\textbf{\textit{#1}}}}
\newcommand{\ErrorTok}[1]{\textcolor[rgb]{0.64,0.00,0.00}{\textbf{#1}}}
\newcommand{\ExtensionTok}[1]{#1}
\newcommand{\FloatTok}[1]{\textcolor[rgb]{0.00,0.00,0.81}{#1}}
\newcommand{\FunctionTok}[1]{\textcolor[rgb]{0.13,0.29,0.53}{\textbf{#1}}}
\newcommand{\ImportTok}[1]{#1}
\newcommand{\InformationTok}[1]{\textcolor[rgb]{0.56,0.35,0.01}{\textbf{\textit{#1}}}}
\newcommand{\KeywordTok}[1]{\textcolor[rgb]{0.13,0.29,0.53}{\textbf{#1}}}
\newcommand{\NormalTok}[1]{#1}
\newcommand{\OperatorTok}[1]{\textcolor[rgb]{0.81,0.36,0.00}{\textbf{#1}}}
\newcommand{\OtherTok}[1]{\textcolor[rgb]{0.56,0.35,0.01}{#1}}
\newcommand{\PreprocessorTok}[1]{\textcolor[rgb]{0.56,0.35,0.01}{\textit{#1}}}
\newcommand{\RegionMarkerTok}[1]{#1}
\newcommand{\SpecialCharTok}[1]{\textcolor[rgb]{0.81,0.36,0.00}{\textbf{#1}}}
\newcommand{\SpecialStringTok}[1]{\textcolor[rgb]{0.31,0.60,0.02}{#1}}
\newcommand{\StringTok}[1]{\textcolor[rgb]{0.31,0.60,0.02}{#1}}
\newcommand{\VariableTok}[1]{\textcolor[rgb]{0.00,0.00,0.00}{#1}}
\newcommand{\VerbatimStringTok}[1]{\textcolor[rgb]{0.31,0.60,0.02}{#1}}
\newcommand{\WarningTok}[1]{\textcolor[rgb]{0.56,0.35,0.01}{\textbf{\textit{#1}}}}
\usepackage{longtable,booktabs,array}
\usepackage{calc} % for calculating minipage widths
% Correct order of tables after \paragraph or \subparagraph
\usepackage{etoolbox}
\makeatletter
\patchcmd\longtable{\par}{\if@noskipsec\mbox{}\fi\par}{}{}
\makeatother
% Allow footnotes in longtable head/foot
\IfFileExists{footnotehyper.sty}{\usepackage{footnotehyper}}{\usepackage{footnote}}
\makesavenoteenv{longtable}
\usepackage{graphicx}
\makeatletter
\def\maxwidth{\ifdim\Gin@nat@width>\linewidth\linewidth\else\Gin@nat@width\fi}
\def\maxheight{\ifdim\Gin@nat@height>\textheight\textheight\else\Gin@nat@height\fi}
\makeatother
% Scale images if necessary, so that they will not overflow the page
% margins by default, and it is still possible to overwrite the defaults
% using explicit options in \includegraphics[width, height, ...]{}
\setkeys{Gin}{width=\maxwidth,height=\maxheight,keepaspectratio}
% Set default figure placement to htbp
\makeatletter
\def\fps@figure{htbp}
\makeatother
\setlength{\emergencystretch}{3em} % prevent overfull lines
\providecommand{\tightlist}{%
  \setlength{\itemsep}{0pt}\setlength{\parskip}{0pt}}
\setcounter{secnumdepth}{-\maxdimen} % remove section numbering
\ifLuaTeX
  \usepackage{selnolig}  % disable illegal ligatures
\fi
\IfFileExists{bookmark.sty}{\usepackage{bookmark}}{\usepackage{hyperref}}
\IfFileExists{xurl.sty}{\usepackage{xurl}}{} % add URL line breaks if available
\urlstyle{same}
\hypersetup{
  pdftitle={CARET\_Lab},
  hidelinks,
  pdfcreator={LaTeX via pandoc}}

\title{CARET\_Lab}
\author{}
\date{\vspace{-2.5em}2023-12-03}

\begin{document}
\maketitle

Package loading

\begin{Shaded}
\begin{Highlighting}[]
\FunctionTok{library}\NormalTok{(caret)}
\end{Highlighting}
\end{Shaded}

\begin{verbatim}
## Loading required package: ggplot2
\end{verbatim}

\begin{verbatim}
## Loading required package: lattice
\end{verbatim}

Load Data

\begin{Shaded}
\begin{Highlighting}[]
\CommentTok{\# attach the iris dataset to the environment}
\FunctionTok{data}\NormalTok{(iris)}
\CommentTok{\# rename the dataset}
\NormalTok{dataset }\OtherTok{\textless{}{-}}\NormalTok{ iris}
\end{Highlighting}
\end{Shaded}

Task1: Create a Validation/Training Dataset You need to split the loaded
dataset into two, 80\% of which we will use to train our models and 20\%
that we will hold back as a validation dataset. Hint: use
createDataPartition function

\begin{Shaded}
\begin{Highlighting}[]
\FunctionTok{set.seed}\NormalTok{(}\DecValTok{43}\NormalTok{)}
\NormalTok{index }\OtherTok{\textless{}{-}} \FunctionTok{createDataPartition}\NormalTok{(dataset}\SpecialCharTok{$}\NormalTok{Species, }\AttributeTok{p =} \FloatTok{0.8}\NormalTok{, }\AttributeTok{list =} \ConstantTok{FALSE}\NormalTok{)}
\NormalTok{train\_data }\OtherTok{\textless{}{-}}\NormalTok{ dataset[index, ]}
\NormalTok{validation\_data }\OtherTok{\textless{}{-}}\NormalTok{ dataset[}\SpecialCharTok{{-}}\NormalTok{index, ]}
\end{Highlighting}
\end{Shaded}

Task2: Summarize Dataset Use skimr library to summarize the dataset

\begin{Shaded}
\begin{Highlighting}[]
\FunctionTok{library}\NormalTok{(skimr)}
\NormalTok{summary }\OtherTok{\textless{}{-}} \FunctionTok{skim}\NormalTok{(dataset)}
\NormalTok{summary}
\end{Highlighting}
\end{Shaded}

\begin{longtable}[]{@{}ll@{}}
\caption{Data summary}\tabularnewline
\toprule\noalign{}
\endfirsthead
\endhead
\bottomrule\noalign{}
\endlastfoot
Name & dataset \\
Number of rows & 150 \\
Number of columns & 5 \\
\_\_\_\_\_\_\_\_\_\_\_\_\_\_\_\_\_\_\_\_\_\_\_ & \\
Column type frequency: & \\
factor & 1 \\
numeric & 4 \\
\_\_\_\_\_\_\_\_\_\_\_\_\_\_\_\_\_\_\_\_\_\_\_\_ & \\
Group variables & None \\
\end{longtable}

\textbf{Variable type: factor}

\begin{longtable}[]{@{}
  >{\raggedright\arraybackslash}p{(\columnwidth - 10\tabcolsep) * \real{0.1728}}
  >{\raggedleft\arraybackslash}p{(\columnwidth - 10\tabcolsep) * \real{0.1235}}
  >{\raggedleft\arraybackslash}p{(\columnwidth - 10\tabcolsep) * \real{0.1728}}
  >{\raggedright\arraybackslash}p{(\columnwidth - 10\tabcolsep) * \real{0.0988}}
  >{\raggedleft\arraybackslash}p{(\columnwidth - 10\tabcolsep) * \real{0.1111}}
  >{\raggedright\arraybackslash}p{(\columnwidth - 10\tabcolsep) * \real{0.3210}}@{}}
\toprule\noalign{}
\begin{minipage}[b]{\linewidth}\raggedright
skim\_variable
\end{minipage} & \begin{minipage}[b]{\linewidth}\raggedleft
n\_missing
\end{minipage} & \begin{minipage}[b]{\linewidth}\raggedleft
complete\_rate
\end{minipage} & \begin{minipage}[b]{\linewidth}\raggedright
ordered
\end{minipage} & \begin{minipage}[b]{\linewidth}\raggedleft
n\_unique
\end{minipage} & \begin{minipage}[b]{\linewidth}\raggedright
top\_counts
\end{minipage} \\
\midrule\noalign{}
\endhead
\bottomrule\noalign{}
\endlastfoot
Species & 0 & 1 & FALSE & 3 & set: 50, ver: 50, vir: 50 \\
\end{longtable}

\textbf{Variable type: numeric}

\begin{longtable}[]{@{}
  >{\raggedright\arraybackslash}p{(\columnwidth - 20\tabcolsep) * \real{0.1842}}
  >{\raggedleft\arraybackslash}p{(\columnwidth - 20\tabcolsep) * \real{0.1316}}
  >{\raggedleft\arraybackslash}p{(\columnwidth - 20\tabcolsep) * \real{0.1842}}
  >{\raggedleft\arraybackslash}p{(\columnwidth - 20\tabcolsep) * \real{0.0658}}
  >{\raggedleft\arraybackslash}p{(\columnwidth - 20\tabcolsep) * \real{0.0658}}
  >{\raggedleft\arraybackslash}p{(\columnwidth - 20\tabcolsep) * \real{0.0526}}
  >{\raggedleft\arraybackslash}p{(\columnwidth - 20\tabcolsep) * \real{0.0526}}
  >{\raggedleft\arraybackslash}p{(\columnwidth - 20\tabcolsep) * \real{0.0658}}
  >{\raggedleft\arraybackslash}p{(\columnwidth - 20\tabcolsep) * \real{0.0526}}
  >{\raggedleft\arraybackslash}p{(\columnwidth - 20\tabcolsep) * \real{0.0658}}
  >{\raggedright\arraybackslash}p{(\columnwidth - 20\tabcolsep) * \real{0.0789}}@{}}
\toprule\noalign{}
\begin{minipage}[b]{\linewidth}\raggedright
skim\_variable
\end{minipage} & \begin{minipage}[b]{\linewidth}\raggedleft
n\_missing
\end{minipage} & \begin{minipage}[b]{\linewidth}\raggedleft
complete\_rate
\end{minipage} & \begin{minipage}[b]{\linewidth}\raggedleft
mean
\end{minipage} & \begin{minipage}[b]{\linewidth}\raggedleft
sd
\end{minipage} & \begin{minipage}[b]{\linewidth}\raggedleft
p0
\end{minipage} & \begin{minipage}[b]{\linewidth}\raggedleft
p25
\end{minipage} & \begin{minipage}[b]{\linewidth}\raggedleft
p50
\end{minipage} & \begin{minipage}[b]{\linewidth}\raggedleft
p75
\end{minipage} & \begin{minipage}[b]{\linewidth}\raggedleft
p100
\end{minipage} & \begin{minipage}[b]{\linewidth}\raggedright
hist
\end{minipage} \\
\midrule\noalign{}
\endhead
\bottomrule\noalign{}
\endlastfoot
Sepal.Length & 0 & 1 & 5.84 & 0.83 & 4.3 & 5.1 & 5.80 & 6.4 & 7.9 &
▆▇▇▅▂ \\
Sepal.Width & 0 & 1 & 3.06 & 0.44 & 2.0 & 2.8 & 3.00 & 3.3 & 4.4 &
▁▆▇▂▁ \\
Petal.Length & 0 & 1 & 3.76 & 1.77 & 1.0 & 1.6 & 4.35 & 5.1 & 6.9 &
▇▁▆▇▂ \\
Petal.Width & 0 & 1 & 1.20 & 0.76 & 0.1 & 0.3 & 1.30 & 1.8 & 2.5 &
▇▁▇▅▃ \\
\end{longtable}

Task3: split input and output It is the time to seperate the input
attributes and the output attributes. call the inputs attributes x and
the output attribute (or class) y.

\begin{Shaded}
\begin{Highlighting}[]
\NormalTok{x }\OtherTok{\textless{}{-}}\NormalTok{ dataset[ , }\DecValTok{1}\SpecialCharTok{:}\DecValTok{4}\NormalTok{]}
\NormalTok{y }\OtherTok{\textless{}{-}}\NormalTok{ dataset}\SpecialCharTok{$}\NormalTok{Species}
\FunctionTok{head}\NormalTok{(x)}
\end{Highlighting}
\end{Shaded}

\begin{verbatim}
##   Sepal.Length Sepal.Width Petal.Length Petal.Width
## 1          5.1         3.5          1.4         0.2
## 2          4.9         3.0          1.4         0.2
## 3          4.7         3.2          1.3         0.2
## 4          4.6         3.1          1.5         0.2
## 5          5.0         3.6          1.4         0.2
## 6          5.4         3.9          1.7         0.4
\end{verbatim}

\begin{Shaded}
\begin{Highlighting}[]
\FunctionTok{head}\NormalTok{(y)}
\end{Highlighting}
\end{Shaded}

\begin{verbatim}
## [1] setosa setosa setosa setosa setosa setosa
## Levels: setosa versicolor virginica
\end{verbatim}

Task4: Train Control for Validation Test

We will use 10-fold crossvalidation to estimate accuracy.

\begin{Shaded}
\begin{Highlighting}[]
\CommentTok{\# Run algorithms using 10{-}fold cross validation}
\NormalTok{control }\OtherTok{\textless{}{-}} \FunctionTok{trainControl}\NormalTok{(}\AttributeTok{method=}\StringTok{"cv"}\NormalTok{, }\AttributeTok{number=}\DecValTok{10}\NormalTok{)}
\NormalTok{metric }\OtherTok{\textless{}{-}} \StringTok{"Accuracy"}
\end{Highlighting}
\end{Shaded}

Task5: Model Training Train 5 different algorithms using `train'
function:

\begin{itemize}
\tightlist
\item
  Linear Discriminant Analysis (LDA)
\item
  Classification and Regression Trees (CART).
\item
  k-Nearest Neighbors (kNN).
\item
  Support Vector Machines (SVM) with a linear kernel.
\item
  Random Forest (RF)
\end{itemize}

\begin{Shaded}
\begin{Highlighting}[]
\FunctionTok{library}\NormalTok{(caret)}
\FunctionTok{library}\NormalTok{(MASS)          }\CommentTok{\# For LDA}
\FunctionTok{library}\NormalTok{(rpart)         }\CommentTok{\# For CART}
\FunctionTok{library}\NormalTok{(class)         }\CommentTok{\# For kNN}
\FunctionTok{library}\NormalTok{(e1071)         }\CommentTok{\# For SVM}
\FunctionTok{library}\NormalTok{(randomForest)  }\CommentTok{\# For RF}
\end{Highlighting}
\end{Shaded}

\begin{verbatim}
## randomForest 4.7-1.1
\end{verbatim}

\begin{verbatim}
## Type rfNews() to see new features/changes/bug fixes.
\end{verbatim}

\begin{verbatim}
## 
## Attaching package: 'randomForest'
\end{verbatim}

\begin{verbatim}
## The following object is masked from 'package:ggplot2':
## 
##     margin
\end{verbatim}

\begin{Shaded}
\begin{Highlighting}[]
\CommentTok{\# Linear Discriminant Analysis (LDA)}
\NormalTok{lda\_model }\OtherTok{\textless{}{-}} \FunctionTok{train}\NormalTok{(x, y, }\AttributeTok{method =} \StringTok{"lda"}\NormalTok{, }\AttributeTok{metric =}\NormalTok{ metric, }\AttributeTok{trControl =}\NormalTok{ control)}
\end{Highlighting}
\end{Shaded}

\begin{Shaded}
\begin{Highlighting}[]
\CommentTok{\# Classification and Regression Trees (CART)}
\NormalTok{cart\_model }\OtherTok{\textless{}{-}} \FunctionTok{train}\NormalTok{(x, y, }\AttributeTok{method =} \StringTok{"rpart"}\NormalTok{, }\AttributeTok{metric =}\NormalTok{ metric, }\AttributeTok{trControl =}\NormalTok{ control)}
\end{Highlighting}
\end{Shaded}

\begin{Shaded}
\begin{Highlighting}[]
\CommentTok{\# k{-}Nearest Neighbors (kNN)}
\NormalTok{knn\_model }\OtherTok{\textless{}{-}} \FunctionTok{train}\NormalTok{(x, y, }\AttributeTok{method =} \StringTok{"knn"}\NormalTok{, }\AttributeTok{metric =}\NormalTok{ metric, }\AttributeTok{trControl =}\NormalTok{ control)}
\end{Highlighting}
\end{Shaded}

\begin{Shaded}
\begin{Highlighting}[]
\CommentTok{\# Support Vector Machines (SVM) with a linear kernel}
\NormalTok{svm\_model }\OtherTok{\textless{}{-}} \FunctionTok{train}\NormalTok{(x, y, }\AttributeTok{method =} \StringTok{"svmLinear"}\NormalTok{, }\AttributeTok{metric =}\NormalTok{ metric, }\AttributeTok{trControl =}\NormalTok{ control)}
\end{Highlighting}
\end{Shaded}

\begin{Shaded}
\begin{Highlighting}[]
\CommentTok{\# Random Forest (RF)}
\NormalTok{rf\_model }\OtherTok{\textless{}{-}} \FunctionTok{train}\NormalTok{(x, y, }\AttributeTok{method =} \StringTok{"rf"}\NormalTok{, }\AttributeTok{metric =}\NormalTok{ metric, }\AttributeTok{trControl =}\NormalTok{ control)}
\end{Highlighting}
\end{Shaded}

Task6: Select the Best Model We now have 5 models and accuracy
estimations for each. We need to compare the models to each other and
select the most accurate. Use resamples function to complete this task

\begin{Shaded}
\begin{Highlighting}[]
\NormalTok{models }\OtherTok{\textless{}{-}} \FunctionTok{list}\NormalTok{(}
  \AttributeTok{LDA =}\NormalTok{ lda\_model,}
  \AttributeTok{CART =}\NormalTok{ cart\_model,}
  \AttributeTok{kNN =}\NormalTok{ knn\_model,}
  \AttributeTok{SVM =}\NormalTok{ svm\_model,}
  \AttributeTok{RF =}\NormalTok{ rf\_model)}

\NormalTok{resamps }\OtherTok{\textless{}{-}} \FunctionTok{resamples}\NormalTok{(models)}
\FunctionTok{summary}\NormalTok{(resamps)}
\end{Highlighting}
\end{Shaded}

\begin{verbatim}
## 
## Call:
## summary.resamples(object = resamps)
## 
## Models: LDA, CART, kNN, SVM, RF 
## Number of resamples: 10 
## 
## Accuracy 
##           Min.   1st Qu.    Median      Mean   3rd Qu. Max. NA's
## LDA  0.9333333 0.9500000 1.0000000 0.9800000 1.0000000    1    0
## CART 0.8000000 0.8833333 0.9333333 0.9266667 0.9833333    1    0
## kNN  0.9333333 0.9500000 1.0000000 0.9800000 1.0000000    1    0
## SVM  0.8666667 0.9500000 1.0000000 0.9733333 1.0000000    1    0
## RF   0.8666667 0.9333333 0.9333333 0.9533333 1.0000000    1    0
## 
## Kappa 
##      Min. 1st Qu. Median Mean 3rd Qu. Max. NA's
## LDA   0.9   0.925    1.0 0.97   1.000    1    0
## CART  0.7   0.825    0.9 0.89   0.975    1    0
## kNN   0.9   0.925    1.0 0.97   1.000    1    0
## SVM   0.8   0.925    1.0 0.96   1.000    1    0
## RF    0.8   0.900    0.9 0.93   1.000    1    0
\end{verbatim}

\begin{Shaded}
\begin{Highlighting}[]
\FunctionTok{dotplot}\NormalTok{(resamps)}
\end{Highlighting}
\end{Shaded}

\includegraphics{caret_lab_files/figure-latex/unnamed-chunk-13-1.pdf}
What was the most accurate model? SVM

Task7: Make Prediction (Confusion Matrix) Now we want to get an idea of
the accuracy of the best model on our validation set. Use `predict' and
confusionMatrix functions to complete this task.

\begin{Shaded}
\begin{Highlighting}[]
\NormalTok{predictions }\OtherTok{\textless{}{-}} \FunctionTok{predict}\NormalTok{(lda\_model, }\AttributeTok{newdata =}\NormalTok{ validation\_data)}
\NormalTok{conf\_matrix }\OtherTok{\textless{}{-}} \FunctionTok{confusionMatrix}\NormalTok{(predictions, validation\_data}\SpecialCharTok{$}\NormalTok{Species)}
\FunctionTok{print}\NormalTok{(conf\_matrix)}
\end{Highlighting}
\end{Shaded}

\begin{verbatim}
## Confusion Matrix and Statistics
## 
##             Reference
## Prediction   setosa versicolor virginica
##   setosa         10          0         0
##   versicolor      0         10         0
##   virginica       0          0        10
## 
## Overall Statistics
##                                      
##                Accuracy : 1          
##                  95% CI : (0.8843, 1)
##     No Information Rate : 0.3333     
##     P-Value [Acc > NIR] : 4.857e-15  
##                                      
##                   Kappa : 1          
##                                      
##  Mcnemar's Test P-Value : NA         
## 
## Statistics by Class:
## 
##                      Class: setosa Class: versicolor Class: virginica
## Sensitivity                 1.0000            1.0000           1.0000
## Specificity                 1.0000            1.0000           1.0000
## Pos Pred Value              1.0000            1.0000           1.0000
## Neg Pred Value              1.0000            1.0000           1.0000
## Prevalence                  0.3333            0.3333           0.3333
## Detection Rate              0.3333            0.3333           0.3333
## Detection Prevalence        0.3333            0.3333           0.3333
## Balanced Accuracy           1.0000            1.0000           1.0000
\end{verbatim}

\end{document}
